\section{アルゴリズム学習教材について}

一般に利用されているアルゴリズム学習教材は, プログラミングを含むものとそうでないものに大別できる. 

\subsection{プログラミングを含まない学習教材}

プログラミングを含まないものには, 知識の習得をメインとする教科書や本などを用いた学習や, コンピュータサイエンスアンプラグド(以下CSアンプラグド)と呼ばれるカードなどの教具を用いて直感的にアルゴリズムを理解することを目的とするものなどがある. 

CSアンプラグドに関しては, 小学校におけるプログラミング学習の必修化などに伴い, 主に小学生や中学生を対象とする低年齢向け教材の研究が活発に行われている. 
また, 低年齢向けであることから高度なアルゴリズムを対象としたものは少なく, 主に二進数やソートアルゴリズムなどの基本となるアルゴリズムの学習を目的とした教材が多い. 

\subsection{プログラミングを含む学習教材}

プログラミングを含むものには, 本などの教材に問題が載っていてそれを解く形式のものや, オンライン上の教材でOJSによる実行環境があり, その場で正誤判定を行えるものなどがある. 
また, 競技プログラマ向けの本であるプログラミングコンテストチャレンジブック\cite{procon challenge}などでは, 実際にWeb上のOJSに掲載されている問題を取り扱うなど, OJSを利用して正誤判定ができる紙媒体の教材もある. 

アルゴ式\cite{algo-method}やpaizaラーニング\cite{paiza}などのウェブ上の教材ではOJSを利用する形式となっている. これは自分の書いたコードの正誤判定を実際の数値を用いて判定することができるため, 人の目による判定よりもより確実で速いことが理由と考えられる. 

% もうちょい弄る予定