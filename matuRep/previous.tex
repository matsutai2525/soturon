\section{関連研究}

関連研究は, アルゴリズムの学習支援に関する研究, オンラインジャッジシステムの活用に関する研究の2種類に分けることができる. 

\subsection{アルゴリズムの学習支援}
コンピュータを用いた初学者に対するアルゴリズム学習支援は, ソースコードを用いるものと用いないものの2種類に分けることができる. 

\subsubsection{ソースコードを用いる学習支援}
新開ら\cite{shinkai2007}は問題を細かい問題に分解していくことによりアルゴリズムの構造を考え, それを基にプログラミングを行う手法を提案している. 
この手法を用いた実験の結果, 制御構造の理解を深めるのに役立つという評価が得られた. 

大城ら\cite{ooshiro2018}はソースコードを基にアニメーションを作成する手法を提案している. 
ソースコードを実行すると, その処理がアニメーションとして実行される. 
アニメーション実行中はソースコード内の対応する行がハイライト表示になり, どこを実行しているのかが分かりやすくなっている. 

\subsubsection{ソースコードを用いない学習支援}
佐藤ら\cite{sato2012}はフローチャートを用いた自習システムを開発している. 
このシステムでは, 問題に対して行う処理をフローチャートで記述し, その正しさを判定したりC言語のソースプログラムに変換することが可能となっている. 

阿部\cite{abe2003}や大森ら\cite{oomori1996}のようにアニメーションを利用した教材も多く作られている. 
また, Algorithm Visualizer\cite{Algorithm Visualizer}やVISUALGO\cite{VISUALGO}のようにアルゴリズムの動作をアニメーションで確認できるWebサイトも存在する. 

\clearpage

\subsection{オンラインジャッジシステムの活用}

\subsubsection{オンラインジャッジシステムとは}
\label{subsec:ojs}
オンラインジャッジシステム(以下OJS)は, 問題に対して予め用意された入力値と出力値の組み合わせを利用して, 提出されたコードをオンライン上で実行し, 自動で正誤判定を行うシステムのことである. 
OJSはAtCoder\cite{AtCoder}やTopCoder\cite{TopCoder}といった競技プログラミングサイトやpaizaラーニング\cite{paiza}などのプログラミングやアルゴリズムの学習サイト等で利用されている. 

また, OnlineJudge2.0\cite{OnlineJudge2.0}やArrow Judge\cite{Arrow Judge}など, オープンソースのOJSも公開されており, 個人でシステムを構築することも容易となっている. 

% 多くのOJSでは多数のプログラミング言語による実行をサポートしており, 利用者は自分の利用しやすい言語を利用して問題を解くことができる. 

\subsubsection{講義における活用}

長尾ら\cite{nagao2016}は採点作業の効率化を目的としてプログラミング演習の講義でOJSを利用している. 
従来の紙媒体を利用した場合と比較した結果, 採点作業の効率化は行えたが学習効果には大きな差は見られなかった. 
% また, 電子媒体ではコピー&ペーストが容易であるため, その対策が必要であるとしている. 

松永\cite{matsunaga2011}はプログラミング演習の再履修講義でOJSを利用している. 
元のプログラミング演習の講義では, 課題に対してプログラミングを行い, そのプログラムを教員が目視で採点する形式をとっている. 
再履修講義では正確なプログラミング能力を徐々に獲得していくことを目的としており, 正確なプログラムを要求するOJSと相性はいいと考えられる. 
結果は, OJSによりプログラミング能力を向上させることのできる学生は一部に留まり, OJSの利用のみでは効果を得ることは難しかった. 


\subsection{問題点}

プログラミング教育におけるOJSの利用は検討されているが, アルゴリズム学習に特化した研究は行われていない. 
また, 従来の講義や課題のスタイルではOJSの導入による学習効率の向上は見られておらず, OJSを利用した際に学習効率のいい教材の開発が必要となっている. 


% 以上のように, 教育におけるOJSの利用は検討されているが, OJSの導入のみによる影響は採点時間等の効率化のみに留まっている. 


% 長尾らはプログラミング演習の講義においてOJSを利用している. 
% OJSを利用した課題と従来のプリントを利用した課題と比較し, 採点作業の効率化は行えたが学習効果には大きな差は見られなかったとしている. 
% また, 電子媒体ではコピー&ペーストが容易であるため, その対策が必要であるとしている. 



% プログラミング内でのアルゴリズム学習タイプ
% アルゴリズム自習のためのフローチャート
% アルゴリズム理解のためのアニメーション
% アルゴリズムの分解(段階的詳細化)
% 講義におけるメモ・クイズの利用
% 問題自動生成 流れ図の利用
