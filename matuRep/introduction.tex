\section{はじめに}

近年, 情報化に伴い大量のデータを扱う機会は増加している. 大量のデータを処理する際には, 全てのパターンを計算することは困難であることが多いため, 計算量を削減するために様々なアルゴリズムが活用されている. 

データを扱うプログラムを書く際に役立つアルゴリズムは数多くある. 
これらのアルゴリズムを活用するためには, アルゴリズム名だけではなくそのアルゴリズムを利用して行える処理や計算量, 実装方法など様々な知識を必要とする. 

アルゴリズムの学習をする際には, 理論のみを学習するのではなく実践も併せて行うことにより, 知識が定着しやすい. 
実際に, ウェブ上にあるアルゴリズム学習教材は, コンテストサイト等を活用して問題を解くような形式となっているものが多い. 

% 問題点
プログラミングを含むアルゴリズム学習教材にはアルゴリズムの実装を学習することがメインの教材が多い. 
また, 本などの媒体の場合は具体的な問題等があってもプログラミング環境の構築や正誤判定などの面で実践が難しい場合がある. 

% 言語の話?

大学の講義でもアルゴリズムを扱った講義はあるが, それらの講義ではアルゴリズムの知識の習得が主であり実際の活用やプログラミングなどを扱うことは少ない. 
また, 講義内でプログラミングを扱う際にも, 利用する言語が講義により異なっており, 利用したことのない言語の場合にはアルゴリズムの学習に加えて言語の文法等の学習コストがかかる可能性がある. 
そのため, 初学者が学習する際にはアルゴリズムの具体的な知識の習得と実践に特化した学習を行う機会は少なくなっている. 

% 目的

そこで本研究では, 講義の課題や教材として利用できる, アルゴリズムの知識を深めた上でプログラミングによる実践を行うプロセスに特化した初学者向けの教材の作成を目的とする. 